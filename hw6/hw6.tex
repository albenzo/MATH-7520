\documentclass[10pt]{article}
\usepackage[utf8]{inputenc}
\usepackage{amscd}
\usepackage{amsmath}
\usepackage{amssymb}
\usepackage{amsthm}
\usepackage{listings}
\usepackage{enumerate}
\usepackage[all,cmtip]{xy}

\textwidth=15cm \textheight=22cm \topmargin=0.5cm \oddsidemargin=0.5cm \evensidemargin=0.5cm

\newcommand{\sk}{\vskip 10mm}
\newcommand{\bb}[1]{\mathbb{#1}}
\newcommand{\ra}{\rightarrow}
\newcommand{\rH}{\widetilde{H}}
\newcommand{\wt}[1]{\widetilde{#1}}
\DeclareMathOperator{\img}{Im}
\DeclareMathOperator{\Tor}{Tor}
\DeclareMathOperator{\Ext}{Ext}
\DeclareMathOperator{\Hom}{Hom}
\DeclareMathOperator{\ord}{ord}
\DeclareMathOperator{\rk}{rk}

\theoremstyle{remark}
\newtheorem{problem}{Problem}
\newtheorem{lemma}{Lemma}[problem]


\newtheorem{tpart}{}[problem]
\newtheorem*{ppart}{}

\begin{document}

\begin{problem}[3.2.15]
  For a fixed coefficient field $F$, define the \textbf{Poincar\'e series}
  of a space $X$ to be the formal power series $p(t)=\sum_ia_it^i$ where $a_i$
  is the dimension of $H^i(X;F)$ as a vector space over $F$, assuming this
  dimension is finite for all $i$. Show that $p(X\times Y)=p(X)p(Y)$. Compute
  the Poincar\'e series for $S^n,\bb{R}P^n,\bb{R}P^\infty,\bb{C}P^n$,
  and $\bb{C}P^\infty$.
\end{problem}

\begin{proof}
  Since we are taking coefficients in a field $F$, there will be no
  torsion on the cohomology groups. As such we can use the K\"unneth formula
  \[
    H^*(X\times Y;F)\cong H^*(X;F)\otimes_{F} H^*(Y;F)
  \]
  For a fixed $n$ this isomorphism gives
  \[
    H^n(X\times Y; F) \cong \bigoplus_{i+j=n}H^i(X;F)\otimes H^j(Y;F)
  \]
  Since the tensor product multiplies dimension this implies that
  \[
    \dim (H^n(X\times Y; F)) \cong \dim\left(\bigoplus_{i+j=n}H^i(X;F)\otimes H^j(Y;F)\right)=\sum_{i+j=n}\left(\dim H^i(X;F)\right) \left(\dim H^j(Y;F)\right)
  \]
  If we let $a_i=\dim (H^i(X;F))$ and $b_i=\dim (H^j(X;F))$ then using the above we get
  \[
    p(X\times Y)= \sum_{n=0}^\infty \sum_{i+j=n}a_i b_jt^n = \left(\sum_{i=0}^\infty a_it^i\right)\left(\sum_{j=0}^\infty b_j t^j\right)=p(X)p(Y)
  \]
  as desired.  
\end{proof}

Using Theorem 3.19 for the projective spaces, we get that the Poincar\'e
polynomials for the spaces above are
\begin{align*}
  p(S^n) &= 1+t^n\\
  p(\bb{R}P^n) &= \sum_0^n t^i\\
  p(\bb{R}P^\infty) &= \sum_0^\infty t^n\\
  p(\bb{C}P^n) &= \sum_0^n t^{2n}\\
  p(\bb{C}P^\infty) &= \sum_0^\infty t^{2n}\\
\end{align*}

\sk

\begin{problem}[3.2.16]
  Show that if $X$ and $Y$ are finite CW complexes such that $H^*(X;\bb{Z})$
  and $H^*(Y;\bb{Z})$ contain no elements of order a power of a given prime $p$,
  then the same is true for $X\times Y$. [Apply Theorem 3.15 with coefficients
  in various fields.]
\end{problem}

\begin{proof}[Proof (Revised):]
  Given a finitely generated abelian group $G$ define $\rk_p(G)$ to be
  the number of occurences of the prime $p$ in the primary decomposition
  $G\cong \bb{Z}^r\bigoplus_{p,k}\bb{Z}_{p^k}$.

  From the Universal Coefficient theorem, if the cohomology groups are
  finitely generated, we get
  \begin{align*}
    H^k(X;\bb{Z})&\cong\Tor(H_{k-1}(X))\oplus\bb{Z}^{\rk(H_k(X))}\\
    H^k(Y;\bb{Z})&\cong\Tor(H_{k-1}(Y))\oplus\bb{Z}^{\rk(H_k(Y))}\\
    H^k(X\times Y;\bb{Z})&\cong\Tor(H_{k-1}(X\times Y))\oplus\bb{Z}^{\rk(H_k(X\times Y))}\\
  \end{align*}

  The values for $\Hom$ and $Ext$ with $\bb{Q}$ and $\bb{Z}_p$ are
  \[
    \begin{array}{ll}
      \Ext(H_{k-1},\bb{Q})\cong 0 & \Hom(H_i,\bb{Q})\cong \bb{Q}^{\rk(H_k)}\\
      \Ext(H_{k-1},\bb{Z}_p)\cong \bb{Z}_p^{\rk_p(H_{k-1})} & \Hom(H_i,\bb{Q})\cong \bb{Q}^{\rk(H_k)+\rk_p(H_k)}\\
    \end{array}
  \]

  Using the universal coefficient theorem once more we get
  \[
    \begin{array}{ll}
      H^k(X;\bb{Q})\cong\bb{Q}^{\rk(H_i(X))} & H^k(X;\bb{Z}_p)\cong\bb{Z}_p^{\rk(H_k(X))+\rk_p(H_k(X))+\rk_p(H_{k-1}(X))}\\
      H^k(Y;\bb{Q})\cong\bb{Q}^{\rk(H_i(Y))} & H^k(Y;\bb{Z}_p)\cong\bb{Z}_p^{\rk(H_k(Y))+\rk_p(H_k(Y))+\rk_p(H_{k-1}(Y))}\\
      H^k(X\times Y;\bb{Q})\cong\bb{Q}^{\rk(H_i(X\times Y))} & H^k(X\times Y;\bb{Z}_p)\cong\bb{Z}_p^{\rk(H_k(X\times Y))+\rk_p(H_k(X\times Y))+\rk_p(H_{k-1}(X\times Y))}\\
    \end{array}
  \]

  Since $X,Y$ are finite CW complexes and we are working with
  field coefficients we can use the Kunn\"eth theorem to get
  \begin{align*}
    H^*(X\times Y;\bb{Q})&\cong H^*(X;\bb{Q})\otimes_{\bb{Z}}H^*(Y;\bb{Q})\\
    H^k(X\times Y;\bb{Q})&\cong \bigoplus_{i+j=k} H^i(X;\bb{Q})\otimes H^j(Y;\bb{Q})\\
    \bb{Q}^{\rk(H_k(X\times Y))}&\cong\bigoplus_{i+j=k} \bb{Q}^{\rk(H_i(X))\rk(H_j(X))}
  \end{align*}
  Which gives the equality
  \[
    \rk(H_k(X\times Y))=\sum_{i+j=k} \rk(H_i(X))\rk(H_j(X))
  \]
  If we repeat the process with $\bb{Z}_p$ coefficients we get the equality
  \begin{align*}
    \rk(H_i(X\times Y))+\rk_p(H_i(X\times Y))+\rk_p(H_{i-1}(X\times Y)) = \sum_{i+j=k}&(\rk(H_k(X))+\rk_p(H_k(X))+\rk_p(H_{k-1}(X)))\\
    \cdot&(\rk(H_k(Y))+\rk_p(H_k(Y))+\rk_p(H_{k-1}(Y)))
  \end{align*}

  Now we begin the proof proper. Assume that $H^*(X;\bb{Z})$ and $H^*(Y;\bb{Z})$
  have no elements of order $p^k$. Since the torsion of the $k$th cohomology
  group comes the torsion of the $(k-1)$ homology group it is clear that for
  a space $Z$ that the cohomology ring has no prime power order elements if,
  and only if, $\rk_p(H_k(Z))=0$ for all $k$. As such our assumption is equivalent
  to $\rk_p(H_k(X))=\rk_p(H_k(Y))=0$ for all $k$.

  We proceed to show that $\rk_p(H_k(X\times Y))=0$ by induction. We start with the
  inductive case. Using the equality from $\bb{Z}_p$ coefficients and placing
  zeros where appropriate we get
  \[
    \rk(H_i(X\times Y))+\rk_p(H_i(X\times Y)) = \sum_{i+j=k}\rk(H_k(X)\rk(H_k(Y))
  \]
  
  Then subtracting the equality from the $\bb{Q}$ coefficients to cancel the first
  terms we get
  \[
    \rk_p(H_k(X\times Y))) = 0
  \]
  which completes the inductive case.

  If $k=0$ we still have
  \[
    \rk(H_0(X\times Y))+\rk_p(H_0(X\times Y))+\rk_p(H_{-1}(X\times Y)) = \rk(H_0(X))\rk(H_0(Y))
  \]
  
  However since $H_{-1}(X\times Y)\cong 0$ we get that
  \[
    \rk_p(H_0(X\times Y))=0
  \]

  This completes the proof that $\rk_P(H_k(X\times Y))=0$ for all $k$ and
  as such we have that $H^*(X\times Y;\bb{Z})$ has no elements of power of $p$
  order.
\end{proof}

\sk

\begin{problem}[3.3.3]
  Show that every covering space of an orientable manifold is an orientable
  manifold.
\end{problem} 

\begin{proof}
  Let $X$ be an orientable manifold and $p:\wt{X}\rightarrow X$ be a covering space.
  Since $X$ is orientable we have a function $x\mapsto \mu_x\in H_n(M|X)$ satisfying
  local consistency. We can then define an orientation on $\wt{X}$ via
  $\wt{x}\mapsto \mu_{p(\wt{x})}$. This will satisfy the local consistency condition
  since $\wt{X}$ is locally homeomorphic to $X$.

  Therefore every covering space of an orientable manifold is orientable.
\end{proof}

\end{document}
%%%%%%%%%%%%%%%%%%%%%%%%%%%%%%%%%%%%%%%%%%%%%%%%%%%%%%%%%%%%%%%%%%%%%%%%%%%%%%%%
\documentclass[10pt]{article}
\usepackage[utf8]{inputenc}
\usepackage{amscd}
\usepackage{amsmath}
\usepackage{amssymb}
\usepackage{amsthm}
\usepackage{listings}
\usepackage{enumerate}
\usepackage[all,cmtip]{xy}

\textwidth=15cm \textheight=22cm \topmargin=0.5cm \oddsidemargin=0.5cm \evensidemargin=0.5cm

\newcommand{\sk}{\vskip 10mm}
\newcommand{\bb}[1]{\mathbb{#1}}
\newcommand{\ra}{\rightarrow}
\newcommand{\rH}{\widetilde{H}}
\DeclareMathOperator{\img}{Im}
\DeclareMathOperator{\Ext}{Ext}
\DeclareMathOperator{\Hom}{Hom}

\theoremstyle{plain}
\newtheorem{problem}{Problem}
\newtheorem{lemma}{Lemma}[problem]

\theoremstyle{remark}
\newtheorem{tpart}{}[problem]
\newtheorem*{ppart}{}

\begin{document}

\begin{problem}[3.1.6a]
  Directly from the definitions, compute the simplicial cohomology groups of
  $S^1\times S^1$ with $\bb{Z}$ and $\bb{Z}_2$ coefficients, using the
  $\Delta$-complex structure given in $\S 2.1$.
\end{problem}

Using the given structure we have for our chain complex
\[
  \xymatrix{
    0 \ar[r] & C_2=\langle U,L\rangle \ar[r] & C_1=\langle a,b,c\rangle \ar[r] & C_0=\langle v\rangle \ar[r] & 0
  }
\]

The value of the boundary map on each of these simplices with $\bb{Z}_2$
coefficients
\begin{align*}
  \partial U&= a+b-c\\
  \partial L&= a+b-c\\
  \partial a&= 0\\
  \partial b&= 0\\
  \partial c&= 0\\
  \partial v&= 0\\
\end{align*}

When we dualize the complex we get
\[
  \xymatrix{
    0 & \ar[l]_{\delta_2} C^2 & \ar[l]_{\delta_1} C^1 & \ar[l]_{\delta_0} C^0 & \ar[l] 0
  }
\]
For each group in the dual complex there are $2^{|C_n|}$ maps determined
by where they send the generators. The maps that generate each are
\begin{align*}
  C^0 &=
       \left\langle
       \left(
       \begin{array}{c}
         v\\
         1
       \end{array}
  \right)
  \right\rangle\\
  C^1 &=
       \left\langle
       \left(
       \begin{array}{ccc}
         a & b & c\\
         1 & 0 & 0
       \end{array}
                 \right),
                 \left(
                 \begin{array}{ccc}
                   a & b & c\\
                   0 & 1 & 0
                 \end{array}
                           \right),
                           \left(
                           \begin{array}{ccc}
                             a & b & c\\
                             0 & 0 & 1
                           \end{array}
                                     \right)
                                     \right\rangle\\
  C^2 &=
       \left\langle
       \left(
       \begin{array}{cc}
         U & L\\
         1 & 0\\
       \end{array}
  \right),
  \left(
  \begin{array}{cc}
    U & L\\
    0 & 1
  \end{array}
        \right)
        \right\rangle\\
\end{align*}

For $H^0(T^2;\bb{Z}_2)$ we need $\ker\delta_0$. However since $\partial$ is the zero map
out of $C_1$ so will be $\delta_0$. As such $H^0(T^2;\bb{Z}_2)\cong\bb{Z}_3$.

For $H^1(T^2;\bb{Z}_2)$ the image of $\delta_0$ is trivial. However for $\ker\delta_1$
this will be exactly the maps that send two generators to $1$ and the zero
map. As such we have
\[
  \ker\delta_1 =
  \left\langle
    \left(
      \begin{array}{ccc}
        a & b & c\\
        1 & 1 & 0
      \end{array}
    \right),
    \left(
      \begin{array}{ccc}
        a & b & c\\
        0 & 1 & 1
      \end{array}
    \right),
    \left(
      \begin{array}{ccc}
        a & b & c\\
        1 & 0 & 1
      \end{array}
    \right)
  \right\rangle\\
  =
  \left\langle
    \left(
      \begin{array}{ccc}
        a & b & c\\
        1 & 0 & 1
      \end{array}
    \right),
    \left(
      \begin{array}{ccc}
        a & b & c\\
        1 & 1 & 0
      \end{array}
    \right)
  \right\rangle\\
\]

Which gives us that $H^1(T^2;\bb{Z}_2)=\bb{Z}_2\oplus\bb{Z}_2$.

Finally for $H^2(T^2;\bb{Z}_2)$ the kernel of $\delta_2$ is everything. As such
the only thing we need to determine is the image of $\delta_1$.
\begin{align*}
  \img \delta_2 &=
                  \left\langle\delta_2
                  \left(
                  \begin{array}{ccc}
                    a & b & c\\
                    1 & 0 & 0
                  \end{array}
                            \right),\delta_2
                            \left(
                            \begin{array}{ccc}
                              a & b & c\\
                              0 & 1 & 0
                            \end{array}
                                      \right),\delta_2
                                      \left(
                                      \begin{array}{ccc}
                                        a & b & c\\
                                        0 & 0 & 1
                                      \end{array}
                                                \right)
                                                \right\rangle\\
                &=
                  \left\langle
                  \left(
                  \begin{array}{ccc}
                    a & b & c\\
                    1 & 0 & 0
                  \end{array}
                            \right)\circ\partial,
                            \left(
                            \begin{array}{ccc}
                              a & b & c\\
                              0 & 1 & 0
                            \end{array}
                                      \right)\circ\partial,
                                      \left(
                                      \begin{array}{ccc}
                                        a & b & c\\
                                        0 & 0 & 1
                                      \end{array}
                                                \right)\circ\partial
                                                \right\rangle\\
                &=
                  \left\langle
                  \left(
                  \begin{array}{cc}
                    U&L\\
                    1&1\\
                  \end{array}
  \right)
  \right\rangle
\end{align*}

Which gives us
\[
  H^2(T^2;\bb{Z}_2)\cong 
  \left\langle
    \left(
      \begin{array}{cc}
        U&L\\
        1&0
      \end{array}
    \right),
    \left(
      \begin{array}{cc}
        U&L\\
        1&0
      \end{array}
    \right)
  \right\rangle
  /
  \left\langle
    \left(
      \begin{array}{cc}
        U&L\\
        1&0\\
      \end{array}
    \right)
    +
    \left(
      \begin{array}{cc}
        U&L\\
        0&1\\
      \end{array}
    \right)
  \right\rangle
  \cong \bb{Z}_2
\]

All other cohomology groups are zero since there are no simplices
of dimension higher than two.

\sk

\begin{problem}[3.1.8a]
  Compute $H^i(S^n;G)$ by induction on $n$ in two ways: using the long exact
  sequence of a pair, and using the Mayer-Vietoris sequence.
\end{problem}

\begin{proof}
  First note that $H^0(S^0;G)\cong G\oplus G$ and in particular $\rH^0(S^0;G)\cong G$ with
  $\rH^k(S^0;G)=0$ for all $k>0$.
  
  We start by using the long exact sequence of the pair in relative
  cohomology using $(D^n,\partial D^n=S^{n-1})$, so $D^n/\partial D^n=S^n$,  slightly thickening
  the boundary so we have a proper excisive couple. Assume that
  $\rH^{n-1}(S^{n-1};G)\cong G$ and that $\rH^k(S^{n-1};G)=0$ for $k\neq n-1$. Now we
  deduce the reduced cohomology of the $n$-sphere.

  For general $k$ we have
  \[
    \xymatrix{
      \cdots \ar[r] & \rH^k(S^n;G)\ar[r] & \rH^k(D^n;G)\ar[r] & \rH^k(S^{n-1};G) \ar[r] & \cdots      
    }
  \]

  For $k\neq n$ this sequence will look like
  \[
    \xymatrix{
      \rH^{k-1}(S^{n-1};G)=0 \ar[r] & \rH^k(S^n;G) \ar[r] & 0=\rH^k(D^n;G)
    }
  \]
  forcing $\rH^k(S^n;G)= 0$ for $k\neq n$. However when $k=n$ we have
  \[
    \xymatrix{
      \rH^{n-1}(D^n;G)=0\ar[r] & \rH^{n-1}(S^{n-1};G)\cong G \ar[r] & \rH^n(S^n;G) \ar[r] & 0=\rH^n(D^n;G)
    }
  \]
  Which gives an isomorphism $\rH^{n-1}(S^{n-1};G)\cong\rH^n(S^n;G)\cong G$.

  \sk

  Next we'll prove the same fact using the Mayer-Vietoris sequence. Break
  up $S^n$ as two copies of $D^n$ with intersection $S^{n-1}$. As before
  assume that $\rH^{n-1}(S^{n-1};G)\cong G$ and $\rH^k(S^{n-1};G)\cong 0$ for $n\neq k$.

  The Mayer-Vietoris sequence with the above decomposition will be
  \[
    \xymatrix{
      \cdots \ar[r] & \rH^k(S^n;G) \ar[r] & \rH^k(D^n;G)\oplus\rH^k(D^n;G) \ar[r] & \rH^k(S^{n-1};G) \ar[r] & \cdots
    }
  \]

  Similarly when $n\neq k$ we get
  \[
    \xymatrix{
      \rH^{k-1}(S^{n-1};G)=0 \ar[r] & \rH^k(S^n;G) \ar[r] & 0=\rH^k(D^n;G)\oplus\rH^k(D^n;G)
    }
  \]
  Once again forcing $\rH^k(S^n;G)=0$ for $n\neq k$. However if $n=k$ we instead get
  \[
    \xymatrix{
     \rH^{n-1}(D^n;G)\oplus\rH^{n-1}(D^n;G)=0 \ar[r] & \rH^{n-1}(S^{n-1};G)\cong G \ar[r] & \rH^n(S^n;G) \ar[r] & 0=\rH^{n}(D^n;G)\oplus\rH^{n}(D^n;G)
    }
  \]
  This gives an isomorphism $\rH^{n-1}(S^{n-1};G)\cong\rH^n(S^n;G)\cong G$.

  From both the above arguments we have determined the reduced cohomology
  of the $n$-sphere and from this we can see that
  \[
    H^k(S^n;G)\cong 
    \left\{
      \begin{array}{lr}
        G& k=0,n\\
        0& \text{otherwise}
      \end{array}
    \right.
  \]
\end{proof}

\sk

\begin{problem}[3.1.9]
  Show that if $f:S^n\rightarrow S^n$ has degree $d$ then
  $f^*:H^n(S^n;G)\rightarrow H^n(S^n;G)$ is multiplication by $d$.
\end{problem}

\begin{proof}
  Let $f:S^n\rightarrow S^n$ be a map of degree $d$. From Hatcher page 196 we have a
  commutative diagram of the form
  \[
    \xymatrix{
      0 \ar[r] & \Ext(H_{n-1}(S^n),G)\ar[r] & H^n(S^n;G) \ar[r]^-h & \Hom(H_n(S^n),G) \ar[r] & 0\\
      0 \ar[r] & \Ext(H_{n-1}(S^n),G)\ar[r]\ar[u]^{(f_*)^*} & H^n(S^n;G) \ar[r]^-h\ar[u]^{f^*} & \Hom(H_n(S^n),G) \ar[r]\ar[u]^{(f_*)^*} & 0
    }
  \]

  Note however that since $H_{n-1}(S^n)=0$ that we also have $\Ext(H_{n-1}(S^n),G)=0$.
  This implies that $h$ is an isomorphism. Moreover we know that $f_*$ on
  the right is multiplication by $d$ as $f$ has degree $d$. Since the hom
  dual of a multiplication map is the same map back we get
  \[
    \xymatrix{
      0 \ar[r] & H^n(S^n;G) \ar[r]^-h & \Hom(H_n(S^n),G) \ar[r] & 0\\
      0 \ar[r] & H^n(S^n;G) \ar[r]^-h\ar[u]^{f^*} & \Hom(H_n(S^n),G) \ar[r]\ar[u]^{\cdot d} & 0
    }
  \]
  as our new commutative diagram. However at this point it is clear that $f^*$ must
  also be multiplication by $d$.

  Therefore if a map has degree $d$ then the induced map on cohomology is
  multiplication by $d$.
\end{proof}

%%%%%%%%%%%%%%%%%%%%%%%%%%%%%%%%%%%%%%%%%%%%%%%%%%%%%%%%%%%%%%%%%%%%%%%%%%%%% 
\end{document}

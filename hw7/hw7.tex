\documentclass[10pt]{article}
\usepackage[utf8]{inputenc}
\usepackage{amscd}
\usepackage{amsmath}
\usepackage{amssymb}
\usepackage{amsthm}
\usepackage{listings}
\usepackage{enumerate}
\usepackage[all,cmtip]{xy}

\textwidth=15cm \textheight=22cm \topmargin=0.5cm \oddsidemargin=0.5cm \evensidemargin=0.5cm

\newcommand{\sk}{\vskip 10mm}
\newcommand{\bb}[1]{\mathbb{#1}}
\newcommand{\ra}{\rightarrow}
\newcommand{\rH}{\widetilde{H}}
\newcommand{\wt}[1]{\widetilde{#1}}
\DeclareMathOperator{\img}{Im}
\DeclareMathOperator{\Ext}{Ext}
\DeclareMathOperator{\Hom}{Hom}

\theoremstyle{remark}
\newtheorem{problem}{Problem}
\newtheorem{lemma}{Lemma}[problem]


\newtheorem{tpart}{}[problem]
\newtheorem*{ppart}{}

\begin{document}

\begin{problem}[3.3.2]
  Show that deleting a point from a manifold of dimension greater than $1$
  does not affect the orientability of the manifold.
\end{problem}

\begin{proof}
  Let $M$ be an $n$-manifold, $n>1$, and let $y$ be a point in $M$. Without loss of
  generality assume that $M$ is connected as we could apply to following reasoning
  to each connected component. From Hatcher Prop. 3.25 $M$ is orientable if,
  and only if, the two-sheeted cover
  \[
    \wt{M}=\{\mu_x|\text{$x\in M$ and $\mu_x$ is a local orientation of $M$ at $x$}\}
  \]
  has two components.

  Suppose that $M$ was orientable. Then $\wt{M}$ has two components. For
  $\wt{M\setminus \{y\}}$ this will be identical to $\wt{M}$ but missing both points
  covering $y$. However since the dimension is greater than one this will
  not affect the number of components. As such $\wt{M\setminus \{y\}}$ has two components
  which implies that $M\setminus\{y\}$ is orientable.

  If $M$ is not orientable, then $\wt{M}$ will not have two components.
  Removing the preimages of $y$ in $\wt{M\setminus\{y\}}$ will not change the number
  of components from that of $\wt{M}$ due to the dimension. Thus $M\setminus \{y\}$
  is not orientable.

  Therefore, for an $n$-manifold with dimension greater than 1, removing a
  single point does not affect the orientability of the manifold.
\end{proof}

\sk

\begin{problem}[3.3.4]
  Given a covering space action of a group $G$ on an orientable manifold $M$
  by orientation-preserving homeomorphisms, show that $M/G$ is also orientable.
\end{problem}

\begin{proof}
  To start let $\mathcal{O}$ be a set of representatives for the
  orbits in $M$ from $G$. Choose an orientation for $M$ such that
  for $x\in \mathcal{O}$ if $x\mapsto\mu_x$ then $g\cdot x\mapsto \mu_{g\cdot x}$ for all $g\in G$
  and we satisfy local consistency. We can do this since the cation of $G$
  is orientation preserving and for each point in the orbit have disjoint
  open neighborhoods.

  From here we can define an orientation on $M/G$ by using the points
  $x\in\mathcal{O}$ and sending them $x\mapsto \mu_x$. This is well defined since
  we were consistent in choice for all points in the orbit and we inherit
  local consistency for the same reason. Thus $M/G$ has an orientation
  and as such is orientable.
\end{proof}

\sk

\begin{problem}[3.3.7]
  For a map $f:M\rightarrow N$ between connected, closed, orientable $n$-manifolds
  with fundamental classes $[M]$ and $[N]$, the \textit{degree} of $f$ is
  defined to be the integer $d$ such that $f_*([M])=d[N]$, so the sign of the
  degree depends on the choice of fundamental classes. Show that for any
  connected closed orientable $n$-manifold $M$ there is a degree $1$ map
  $M\rightarrow S^n$.
\end{problem}

\begin{proof}
  Let $p$ be a point in a connected, closed, and oriented $n$-manifold $M$.
  Then $p$ has an open neighborhood $U$ homeomorphic to $\bb{R}^n$. Let $q$
  be a point in $S^n$. Define a map $f:M\rightarrow S^n$ by mapping $U$ homeomorphically
  to $S^n\setminus\{q\}\cong \bb{R}^n$, preserving orientation,
  and mapping $M\setminus U$ to $q$. This map is continuous since an open set in $S^n$
  not containing $q$ will have preimage an open set in $U$ and a set containing
  $q$ will have preimage with complement a closed neighborhood of $p$.

  Thus $f$ is a continuous map from $M$ to $S^n$. Now we show that it has
  degree $1$. Using local degree procedure from Hatcher pg. 136, and the fact
  that we mapped $U$ homeomorphically to $S^n\setminus\{q\}$ preserving orientation, we
  can deduce that the local degree of $f$ at $-q$ is $1$ as
  $f^{-1}(-q)=\{p\}$ and the map $f_*:H_n(U,U\setminus p)\rightarrow H_n(S^n\setminus \{q\},S^n\setminus\{q,-q\})$ is
  the identity map. Since $p$ is the only point in the preimage of $-q$ the
  degree of $f$ is exactly 1.
\end{proof}

\end{document}
%%%%%%%%%%%%%%%%%%%%%%%%%%%%%%%%%%%%%%%%%%%%%%%%%%%%%%%%%%%%%%%%%%%%%%%%%%%%%%%%
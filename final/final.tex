\documentclass[10pt]{article}
\usepackage[utf8]{inputenc}
\usepackage{amscd}
\usepackage{amsmath}
\usepackage{amssymb}
\usepackage{amsthm}
\usepackage{listings}
\usepackage{enumerate}
\usepackage[all,cmtip]{xy}

\textwidth=15cm \textheight=22cm \topmargin=0.5cm \oddsidemargin=0.5cm \evensidemargin=0.5cm

\newcommand{\sk}{\vskip 10mm}
\newcommand{\bb}[1]{\mathbb{#1}}
\newcommand{\ra}{\rightarrow}
\newcommand{\rH}{\widetilde{H}}
\newcommand{\wt}[1]{\widetilde{#1}}
\DeclareMathOperator{\img}{Im}
\DeclareMathOperator{\Ext}{Ext}
\DeclareMathOperator{\Hom}{Hom}

\theoremstyle{remark}
\newtheorem{problem}{Problem}
\newtheorem{lemma}{Lemma}[problem]


\newtheorem{tpart}{}[problem]
\newtheorem*{ppart}{}

\begin{document}

\begin{problem}[3.3.1]
  Show that there exist nonorientable $1$-dimensional manifolds if the Hausdorff
  condition is dropped form the definition of a manifold.
\end{problem}

\begin{proof}
  We construct a tear drop space shown below as the quotient
  \[
    (0,1] \times\{0,1\}/\sim
  \]
  where $(x,0)\sim(x,1)$ if $x<\frac{1}{2}$ and $(1,0)\sim (1,1)$. Each
  point in this space has a neighborhood homeomorphic to $\bb{R}$ since
  we had $x<\frac{1}{2}$ instead of $x\leq\frac{1}{2}$ in the relation.
  The left chunk is homeomorphic to an open interval, as is the right chunk.
  The two points $(1/2,0)$ and $(1/2,1)$ also have open neighborhoods
  homeomorphic to $\bb{R}$ although they cannot be separated.

  The picture below demonstrates the nonorientability of the space.
  \vskip1in
\end{proof}

\sk

\begin{problem}[3.3.11]
  If $M_g$ denotes the closed orientable surface of genus $g$, show
  that degree $1$ maps $M_g\rightarrow M_h$ exist iff $g\geq h$.
\end{problem}

\begin{proof}
  Suppose that $g\geq h$. Decompose $M_g$ as $M_h\#M_{g-h}$ where
  $\#$ is the connected sum. Then map the $M_h$ component of $M_g$
  with an orientation preserving homeomorphism to $M_h$ and map
  $M_{g-h}$ to a point. This will be a degree 1 map by local degree
  of any point aside from the one that $M_{g-h}$ is sent to.

  On the other hand suppose that $g< h$ and there existed
  a degree $1$ map $M_g\rightarrow M_h$. Then by one of the problems
  from the last homework we would have a surjective map on the
  first homologies $\bb{Z}^{2g}\rightarrow \bb{Z}^{2h}$. However this is a
  contradiction since $g<h$.

  Therefore a degree 1 map only exists if $g\geq h$.
\end{proof}

\sk

\begin{problem}[3.3.16]
  Show that
  $(\alpha\smallfrown \varphi)\smallfrown \psi=\alpha\smallfrown (\varphi\smallsmile \psi)$
  for all $\alpha\in C_k(X;R)$, $\varphi\in C^l(X;R)$, and $\psi\in C^m(X;R)$.
  Deduce that cap product makes $H_*(X;R)$ a right $H^*(X;R)$-module.
\end{problem}

\begin{proof}
  
\end{proof}

\sk

\begin{problem}[3.3.17]
  Show that a direct limit of exact sequences is exact. More generally show that
  homology commutes with direct limits: If $\{C_\alpha,f_{\alpha\beta}\}$ is a
  directed system of chain complexes, with the maps
  $f_{\alpha\beta}:C_\alpha\rightarrow C_\beta$ chain maps,
  then $H_n(\varinjlim C_\alpha)=\varinjlim H_n(C_\alpha)$.
\end{problem}

\begin{proof}
  
\end{proof}

\sk

\begin{problem}[3.3.20]
  Show that $H_c^0(X;G)=0$ if $X$ is path-connected and noncompact.
\end{problem}

\begin{proof}
  A $0$-cochain is a cocycle if it is constant on each
  path component. So if $\varphi\in\ker\delta:\Delta_c^0(X;G)\rightarrow\Delta_c^1(X;G)$ then it must be a constant
  function since $X$ is path connected. However since $\varphi$ has compact
  support and $X$ is non-compact it must be that $\varphi\equiv 0$. Since
  $\Delta_c^0(X;G)\cong H_c^0(X;G)$ it follows that $H_c^0(X;G)\cong 0$.
\end{proof}

\sk

\begin{problem}[3.3.25]
  Show that if a closed orientable manifold $M$ of dimension $2k$ has
  $H_{k-1}(M;\bb{Z})$ torsion-free, then $H_k(M;\bb{Z})$ is also torsion-free.
\end{problem}

\begin{proof}
  By Poincar\'e duality we have that $H_k(M;\bb{Z})\cong H^k(M;\bb{Z})$.
  By 3.3 from Hatcher $H^k$ gets its free component form $H_k(M;\bb{Z})$ and
  its torsion component from $H_{k-1}(M;\bb{Z})$. Since $H_{k-1}(M;\bb{Z})$ is
  torsion free, so to is $H_k(M;\bb{Z})$.
\end{proof}

\sk

\begin{problem}[3.3.32]
  Show that a compact manifold does not retract onto its boundary.
\end{problem}

\begin{proof}
  Let $M$ be a compact $n$-manifold with boundary $\partial M$. Suppose that
  there was a retract of $M$ onto its boundary. Then the map induced
  by inclusion $i_*$ is injective. The LES of the pair gives us
  \[
    \xymatrix{
      0 \ar[r] & 
    }
  \]
\end{proof}

\sk

\begin{problem}[3.3.33]
  Show that if $M$ is a compact contractible $n$-manifold then $\partial M$ is a
  homology $(n-1)$-sphere.
\end{problem}

\begin{proof}
  Since $M$ is contractible $H_i(M)\cong 0$ for $i>0$. The the LES of the pair in
  homology gives an isomorphism
  \[
    \xymatrix{
      0 \ar[r] & H_i(M,\partial M) \ar[r] & H_{i-1}(\partial M) \ar[r] & 0
    }
  \]
  By Lefschetz duality $H_i(M,\partial M)\cong H^{n-i}(M)$. These two isomorphisms together
  along with the fact that $H^0(M)\cong \bb{Z}$ gives us that
  \[
    H_i(\partial M) \cong
    \left\{
      \begin{array}{lr}
        \bb{Z} & i=n-1,0\\
        0 & \text{otherwise}
      \end{array}
    \right.
  \]
  Which makes $\partial M$ a homology $(n-1)$-sphere.
\end{proof}

\end{document}
%%%%%%%%%%%%%%%%%%%%%%%%%%%%%%%%%%%%%%%%%%%%%%%%%%%%%%%%%%%%%%%%%%%%%%%%%%%%%%%%
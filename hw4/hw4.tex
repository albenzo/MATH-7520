\documentclass[10pt]{article}
\usepackage[utf8]{inputenc}
\usepackage{amscd}
\usepackage{amsmath}
\usepackage{amssymb}
\usepackage{amsthm}
\usepackage{listings}
\usepackage{enumerate}
\usepackage[all,cmtip]{xy}

\textwidth=15cm \textheight=22cm \topmargin=0.5cm \oddsidemargin=0.5cm \evensidemargin=0.5cm

\newcommand{\sk}{\vskip 10mm}
\newcommand{\bb}[1]{\mathbb{#1}}
\newcommand{\rH}{\widetilde{H}}
\DeclareMathOperator{\img}{Im}
\DeclareMathOperator{\Ext}{Ext}
\DeclareMathOperator{\Hom}{Hom}

\theoremstyle{remark}
\newtheorem{problem}{Problem}
\newtheorem{lemma}{Lemma}[problem]


\newtheorem{tpart}{}[problem]
\newtheorem*{ppart}{}

\begin{document}

\begin{problem}[3.2.1]
  Assuming as known the cup product structure on the torus $S^1\times S^1$,
  compute the cup product structure in $H^*(M_g)$ for $M_g$ the closed
  orientable surface of genus $g$ by using the quotient map from $M_g$
  to a wedge sum of $g$ tori.
\end{problem}

Let $\alpha_i,\beta_i$ denote the first cohomology generators and
let $\gamma_i$ denote the generator for the second cohomology group for the $i$th
tori in the wedge sum. Also let $\gamma$ be the lone generator for $H^2(M_g)$.
Note that since the homology groups of the torus are
free that $\Ext(H_k(T^2),\bb{Z})$ will be zero and as such the cohomology groups are the hom duals
of the homology groups. Let $a_i,b_i,c_i$ be the dual basis elements to $\alpha_i,\beta_i,$ and
$\gamma_i$ respectively. The induced map $q_*:H^k(M_g)\rightarrow H^k(\bigvee_i T^2)$ will act on the basis elements of
the homology groups as
\[
  q_*(a_i)=a_i,\quad q_*(b_9)=b_i,\quad q_*(c)=\sum_i c_i
\]
where $c$ is the lone generator of $H_2(M_g)$. By the universal coefficient theorem
$q^*$ is the hom dual of $q_*$. As such we can deduce the values of $q^*$ as
\[
  q^*(\alpha_8)=\alpha_8,\quad q^*(\beta_i),\quad q^*(\gamma_i)= \gamma
\]

Since the cohomology ring of the wedge sum is the product of the cohomology rings
we can determine the cup product structure using what we know of $\prod_i H^*(T^2)$ and
$q^*$. The cup product structure of $\prod_i H^i(T^2)$ is
\begin{align*}
  \alpha_i\smile \beta_j &= 0 & i\neq j\\
  \alpha_i\smile \beta_i &= \gamma_i\\
  \beta_j\smile \alpha_j &= -\gamma_j\\
  \beta_j\smile \alpha_i &= 0 & i\neq j\\
\end{align*}

Then using the quotient map we get
\begin{align*}
  q^*(\alpha_i)\smile q^*(\beta_j) &= q^*(\alpha_i\smile\beta_j) = 0 & i\neq j\\
  q^*(\alpha_i)\smile q^*(\beta_i) &= q^*(\alpha_i\smile\beta_i) = \gamma_i\\
  q^*(\beta_j)\smile q^*(\alpha_j) &= q^*(\beta_j\smile\alpha_j)= -\gamma_j\\
  q^*(\beta_j)\smile q^*(\alpha_i) &= q^*(\beta_j\smile\alpha_i) = 0 & i\neq j\\
\end{align*}

Which gives us the full cup product structure for $H^*(M_g)$.

\sk

\begin{problem}[3.2.2]
  Using the cup product
  $H^k(X,A;R)\times H^\ell(X,B;R)\rightarrow H^{k+\ell}(X,A\cup B;R)$, show that
  if $X$ is the union of contractible open subsets $A$ and $B$, then all cup
  products of positive-dimensional classes in $H^*(X;R)$ are zero. This applies
  in particular if $X$ is a suspension. Generalize to the situation that $X$ is
  a union of $n$ contractible open subsets, to show that the $n$-fold cup
  products of positive dimensional classes are zero.
\end{problem}

\begin{proof}
  First note that since $A$ is contractible we get an isomorphism
  \[
    \xymatrix{
      0 \ar[r] & H^k(X,A;R) \ar[r] & H^n(X;R) \ar[r] & 0
    }
  \]
  Similarly we also get an isomorphism with $H^k(X,B;R)$ and $H^k(X;R)$.

  Using this with the naturality of the cup product we get a commutative
  diagram
  \[
    \xymatrix{
      H^k(X,A;R)\times H^\ell(X,B;R) \ar[r]^-\smile \ar[d]^{\cong }& H^{k+\ell}(X,A\cup B;R)\cong 0 \ar[d]\\
      H^k(X;R)\times H^\ell(X;R) \ar[r]^-\smile & H^{k+l}(X;R)
    }
  \]
  However since this map factors through zero it must be the case that
  the cup product is zero for positive dimensions.

  In the case where $X=\bigcup_i A_i$ we still have the same isomorphisms as before.
  As such our new diagram is
  \[
    \xymatrix{
      \prod_i H^{k_i}(X,A;R) \ar[r]^-{\smile} \ar[d]^{\cong}& H^{\Sigma_ik_i}(X,\bigcup A_i;R)\cong 0 \ar[d]\\
      \prod_i H^{k_i}(X;R) \ar[r]^{\smile} & H^{\Sigma_i k_i}(X;R)
    }
  \]
  which gives us zero on the cup product for positive dimensions via the same
  reasoning as above.
\end{proof}

\sk

\begin{problem}[3.2.4]
  Apply the Lefschetz fixed point theorem to show that every map
  $f:\bb{C}P^n\rightarrow\bb{C}P^n$ has a fixed point if $n$ is even, using
  the fact that $f^*:H^*(\bb{C}P^n;\bb{Z})\rightarrow H^*(\bb{C}P^n;\bb{Z})$
  is a ring homomorphism. When $n$ is odd show there is a fixed point unless
  $f^*(\alpha)=-\alpha$, for $\alpha$ a generator of $H^2(\bb{C}P^n;\bb{Z})$.
  [See Exercise 3 in \S 2.C for an example of a map without fixed points in
  this exceptional case.]
\end{problem}

\begin{proof}
  Recall that $H^*(\bb{C}P^n;\bb{Z})\cong\bb{Z}(\alpha)/(\alpha^{n+1})$ and that
  $H^k(\bb{C}P^n;\bb{Z})$ is $\bb{Z}$ for even dimensions and zero otherwise. For
  $f^*:H^2(\bb{C}P^n;\bb{Z})\rightarrow H^2(\bb{C}P^n;\bb{Z})$ this is an endomorphism on the integers.
  As such it must be of the form $f^*(\alpha)=c\alpha$ for some $c\in\bb{Z}$. Moreover since
  $f^*$ is also an endomorphism on the cohomology ring it must be that
  $f^*(\alpha^m)=c^m\alpha^m$. Since the $\Ext$ term in the universal coefficient theorem
  will be zero for $\bb{C}P^n$ we have that $f^*$ is the hom dual of $f_*$.
  Since multiplication maps are unaffected by hom dual the Lefschetz number of
  $f$ is
  \[
    \tau(f)= 1+c+\cdots+c^n
  \]

  The only possible rational roots, and thus integer roots, of $\tau(f)$ are $\pm 1$.
  Since all the coefficients are positive $1$ is not a root. If $n$ is even $-1$
  cannot be a root as well since there are an odd number of terms.
  If $n$ is odd and $f^*(\alpha)\neq-\alpha$ (i.e. $c\neq -1$), then $\tau(f)$ will have no roots.
  Therefore by the Lefschetz fixed point theorem a map $f:\bb{C}P^n\rightarrow\bb{C}P^n$
  has a fixed point if $n$ is even or if $n$ is odd and $f(\alpha)\neq -\alpha$.
\end{proof}

\end{document}
%%%%%%%%%%%%%%%%%%%%%%%%%%%%%%%%%%%%%%%%%%%%%%%%%%%%%%%%%%%%%%%%%%%%%%%%%%%%%%%%
\documentclass[10pt]{article}
\usepackage[utf8]{inputenc}
\usepackage{amscd}
\usepackage{amsmath}
\usepackage{amssymb}
\usepackage{amsthm}
\usepackage{listings}
\usepackage{enumerate}
\usepackage[all,cmtip]{xy}

\textwidth=15cm \textheight=22cm \topmargin=0.5cm \oddsidemargin=0.5cm \evensidemargin=0.5cm

\newcommand{\sk}{\vskip 10mm}
\newcommand{\bb}[1]{\mathbb{#1}}
\newcommand{\ra}{\rightarrow}
\newcommand{\rH}{\widetilde{H}}
\DeclareMathOperator{\img}{Im}
\DeclareMathOperator{\Ext}{Ext}
\DeclareMathOperator{\Hom}{Hom}

\theoremstyle{remark}
\newtheorem{problem}{Problem}
\newtheorem{lemma}{Lemma}[problem]


\newtheorem{tpart}{}[problem]
\newtheorem*{ppart}{}

\begin{document}

\begin{problem}[3.2.1]
  Assuming as known the cup product structure on the torus $S^1\times S^1$,
  compute the cup product structure in $H^*(M_g)$ for $M_g$ the closed
  orientable surface of genus $g$ by using the quotient map from $M_g$
  to a wedge sum of $g$ tori.
\end{problem}

\begin{proof}
  
\end{proof}

\sk

\begin{problem}[3.2.2]
  Using the cup product
  $H^k(X,A:R)\times H^\ell(X,B;R)\rightarrow H^{k+\ell}(X,A\cup B;R)$, show that
  if $X$ is the union of contractible open subsets $A$ and $B$, then all cup
  products of positive-dimensional classes in $H^*(X;R)$ are zero. This applies
  in particular if $X$ is a suspension. Generalize to the situation that $X$ is
  a union of $n$ contractible open subsets, to show that the $n$-fold cup
  products of positive dimensional classes are zero.
\end{problem}

\begin{proof}
  
\end{proof}

\sk

\begin{problem}[3.2.4]
  Apply the Lefschetz fixed point theorem to show that every map
  $f:\bb{C}P^n\rightarrow\bb{C}P^n$ has a fixed point if $n$ is even, using
  the fact that $f^*:H^*(\bb{C}P^n;\bb{Z})\rightarrow H^*(\bb{C}P^n;\bb{Z})$
  is a ring homomorphism. When $n$ is odd show there is a fixed point unless
  $f^*(\alpha)=-\alpha$, for $\alpha$ a generator of $H^2(\bb{C}P^n;\bb{Z})$.
  [See Exercise 3 in \S 2.C for an example of a map without fixed points in
  this exceptional case.]
\end{problem}

\begin{proof}
  
\end{proof}

\end{document}
%%%%%%%%%%%%%%%%%%%%%%%%%%%%%%%%%%%%%%%%%%%%%%%%%%%%%%%%%%%%%%%%%%%%%%%%%%%%%%%%
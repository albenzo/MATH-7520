\documentclass[10pt]{article}
\usepackage[utf8]{inputenc}
\usepackage{amscd}
\usepackage{amsmath}
\usepackage{amssymb}
\usepackage{amsthm}
\usepackage{listings}
\usepackage{enumerate}
\usepackage[all,cmtip]{xy}

\textwidth=15cm \textheight=22cm \topmargin=0.5cm \oddsidemargin=0.5cm \evensidemargin=0.5cm

\newcommand{\sk}{\vskip 10mm}
\newcommand{\bb}[1]{\mathbb{#1}}
\newcommand{\ra}{\rightarrow}
\newcommand{\rH}{\widetilde{H}}
\newcommand{\wt}[1]{\widetilde{#1}}
\DeclareMathOperator{\img}{Im}
\DeclareMathOperator{\Ext}{Ext}
\DeclareMathOperator{\Hom}{Hom}

\theoremstyle{remark}
\newtheorem{problem}{Problem}
\newtheorem{lemma}{Lemma}[problem]


\newtheorem{tpart}{}[problem]
\newtheorem*{ppart}{}

\begin{document}

\begin{problem}[3.3.8]
  For a map $f:M\rightarrow N$ between connected closed orientable
  $n$-manifolds, suppose there is a ball $B\subset N$ such that $f^{-1}(B)$ is
  the disjoint union of balls $B_i$ each mapped homeomorphically by $f$ onto
  $B$. Show the degree of $f$ is $\sum_i\epsilon_i$ where $\epsilon_i$ is
  $+1$ or $-1$ according to whether $f:B_i\rightarrow B$ preserves or reverses
  local orientations induced from given fundamental classes $[M]$ and $[N]$.
\end{problem}

\begin{proof}
  Let $y\in B$ and $\{x_1,\ldots,x_m\}=f^{-1}(y)$ where $x_i\in B_i$. Note that there must
  be finitely many $B_i$s or the sum in the problem is not well defined. We then
  have the following adaption of the commutative diagram from the proof of Prop. 2.30 in Hatcher
  \[
    \xymatrix{
      & H_n(B_i,B_i\setminus x_i) \ar[r]^{f_*} \ar[dl]^\cong \ar[d]^{k_i}& H_n(B,B\setminus y) \ar[d]^{\cong}\\
      H_n(M,M\setminus x_i) & H_n(M,M\setminus f^{-1}(y)) \ar[l]^{p_i} \ar[r]^{f_*} & H_n(N,N\setminus y)\\
      & H_n(M) \ar[ul]^{\cong} \ar[u]^j \ar[r]^{f_*} & H_n(N) \ar[u]^\cong \\
    }
  \]
  As in the proposition the upper two arrows come from excision. The lower
  two isomorphisms are isomorphisms by Theorem 3.26 and come from the LES of the
  pair. The local degree at each $x_i$ will be $\pm 1$ as they are homeomorphisms.

  Since this diagram commutes and is identical to the one used to prove Prop. 2.30
  except with $M$ and $N$ instead of $S^n$s we can conclude from the proof of prop. 2.30
  that $\deg f= \sum_i \epsilon_i$.
\end{proof}

\sk

\begin{problem}[3.3.9]
  Show that a $p$-sheeted covering space projection $M\rightarrow N$ has
  degree $\pm p$, when $M$ and $N$ are connected closed orientable manifolds.
\end{problem}

\begin{proof}
  Since covering space projections are local homeomorphisms we can apply the previous
  problem to this by considering a point $y\in N$ and its $p$-preimages. As such
  it will suffice to show that the local degree of each point in $f^{-1}(y)$
  agrees.

  Suppose that not all of the local degrees agree for some subset of $N$. Then
  partition $M$ into $M_+$ and $M_-$ denoting the points of $M$ where $f$ preserves
  and reverses local orientation respectively. It's clear from the definition
  that $M_+\cap M_-=\emptyset$. Moreover both $M_+$ and $M_-$ are open as given a point $x\in M$
  $f(x)$ has an open neighborhood $U$ that is oriented and each disjoint sheet in
  $f^{-1}(U)$ must either have orientation preserved or reversed. Thus $M_+$ and
  $M_-$ form a partition of $M$ which contradicts our assumption that $M$ was
  connected.

  Therefore given a point $y\in N$ the local degree must be $\pm p$ and by the
  previous problem the degree of $f$ is then $\pm p$.
\end{proof}

\sk

\begin{problem}[3.3.10]
  Show that for a degree $1$ map $f:M\rightarrow N$ of connected closed
  orientable manifolds, the induced map $f_*:\pi_1 M\rightarrow \pi_1 N$ is
  surjective, hence also $f_*:H_1(M)\rightarrow H_1(N)$. [Lift $f$ to the
  covering space $\wt{N}\rightarrow N$ corresponding to the subgroup
  $\img f_*\subset\pi_1 N$, then consider the two cases that this covering is
  finite sheeted or infinite sheeted.]
\end{problem}

\begin{proof}
  Let $\wt{N}$ be the covering space corresponding to the subgroup
  $\img f_* \subset\pi_1(N,*)$ with covering map $p$. Then we can lift $f$
  to $\wt{f}:M\rightarrow\wt{N}$. Now since degree is multiplicative under composition
  and covering spaces are manifolds we have that $\deg f=\deg \wt{f}\cdot \deg p$.

  The case where $\wt{N}$ is an infinite sheeted cover cannot occur since
  as $\wt{N}$ would not be compact forcing the isomorphism $f_*:H_n(M)\rightarrow H_n(N)$
  to factor through $H_n(\wt{N})\cong 0$.
  
  In the case where $\wt{f}$ has a finite number of sheets we know from the previous
  problem that the degree is the number of sheets. Using the same degree relation
  and the fact that degrees are integers the only possibilities when $\deg f=1$
  is that $\deg\wt{f}=\deg p=\pm 1$. However this also implies that $\wt{N}$ is
  a one-sheeted cover and as such $\wt{N}$ is homeomorphic to $N$. From this we
  can conclude that $\img f_*=\pi_1(N,*)$ and that $f_*:\pi_1(M,*)\rightarrow\pi_1(N,*)$ must be
  surjective. This also implies that the induced map $f_*$ on the $1$st homology
  is surjective as $H_1$ is the abelianization of $\pi_1$.
\end{proof}
\end{document}
%%%%%%%%%%%%%%%%%%%%%%%%%%%%%%%%%%%%%%%%%%%%%%%%%%%%%%%%%%%%%%%%%%%%%%%%%%%%%%%%
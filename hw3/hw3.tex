\documentclass[10pt]{article}
\usepackage[utf8]{inputenc}
\usepackage{amscd}
\usepackage{amsmath}
\usepackage{amssymb}
\usepackage{amsthm}
\usepackage{listings}
\usepackage{enumerate}
\usepackage[all,cmtip]{xy}

\textwidth=15cm \textheight=22cm \topmargin=0.5cm \oddsidemargin=0.5cm \evensidemargin=0.5cm

\newcommand{\sk}{\vskip 10mm}
\newcommand{\bb}[1]{\mathbb{#1}}
\newcommand{\ra}{\rightarrow}
\newcommand{\rH}{\widetilde{H}}
\DeclareMathOperator{\img}{Im}
\DeclareMathOperator{\Ext}{Ext}
\DeclareMathOperator{\Hom}{Hom}

\theoremstyle{plain}
\newtheorem{problem}{Problem}
\newtheorem{lemma}{Lemma}[problem]

\theoremstyle{remark}
\newtheorem{tpart}{}[problem]
\newtheorem*{ppart}{}

\begin{document}

\begin{problem}[3.1.10]
  For the lens space $L_m(\ell_1,\ldots,\ell_n)$ defined in Example 2.43,
  compute the cohomology groups using the cellular cochain complex
  and taking coefficients in $\bb{Z},\bb{Q},\bb{Z}_m,$ and $\bb{Z}_p$
  for $p$ prime. Verify that the answers agree with those given by the
  universal coefficient theorem.
\end{problem}

\begin{proof}
  
\end{proof}

\sk

\begin{problem}[3.1.11]
  Let $X$ be a Moore space $M(\bb{Z}_m,n)$ obtained from $S^n$ by attaching
  a cell $e^{n+1}$ by a map of degree $m$.
  \begin{itemize}
  \item[(a)] Show that the quotient map $X\rightarrow X/S^n=S^{n+1}$ induces the
    trivial map on $\rH_i(-;\bb{Z})$ for all $i$, but not on
    $H^{n+1}(-;\bb{Z})$. Deduce that the splitting in the universal coefficient
    theorem for cohomology cannot be natural.
  \item[(b)] Show that the inclusion $S^n\hookrightarrow X$ induces the trivial
    map on $\rH^i(-;\bb{Z})$ for all $i$, but not on $H_n(-;\bb{Z})$.
  \end{itemize}
\end{problem}

\begin{proof}
  
\end{proof}

\end{document}
%%%%%%%%%%%%%%%%%%%%%%%%%%%%%%%%%%%%%%%%%%%%%%%%%%%%%%%%%%%%%%%%%%%%%%%%%%%%%%%%
\documentclass[10pt]{article}
\usepackage[utf8]{inputenc}
\usepackage{amscd}
\usepackage{amsmath}
\usepackage{amssymb}
\usepackage{amsthm}
\usepackage{listings}
\usepackage{enumerate}
\usepackage[all,cmtip]{xy}

\textwidth=15cm \textheight=22cm \topmargin=0.5cm \oddsidemargin=0.5cm \evensidemargin=0.5cm

\newcommand{\sk}{\vskip 10mm}
\newcommand{\bb}[1]{\mathbb{#1}}
\newcommand{\ra}{\rightarrow}
\newcommand{\rH}{\widetilde{H}}
\DeclareMathOperator{\img}{Im}
\DeclareMathOperator{\Ext}{Ext}
\DeclareMathOperator{\Hom}{Hom}

\theoremstyle{remark}
\newtheorem{problem}{Problem}
\newtheorem{lemma}{Lemma}[problem]


\newtheorem{tpart}{}[problem]
\newtheorem*{ppart}{}

\begin{document}

\begin{problem}[3.1.10]
  For the lens space $L_m(\ell_1,\ldots,\ell_n)$ defined in Example 2.43,
  compute the cohomology groups using the cellular cochain complex
  and taking coefficients in $\bb{Z},\bb{Q},\bb{Z}_m,$ and $\bb{Z}_p$
  for $p$ prime. Verify that the answers agree with those given by the
  universal coefficient theorem.
\end{problem}

\begin{proof}
  We know Hatcher that the cellular chain complex for a lens space
  $L_m(\ell_1,\ldots,l_n)$ is
  \[
    \xymatrix{
      0 \ar[r] & \bb{Z} \ar[r]^0 & \bb{Z} \ar[r]^m & \bb{Z} \ar[r]^0 & \cdots \ar[r]^0 & \bb{Z} \ar[r]^m & \bb{Z} \ar[r]^0 & \bb{Z} \ar[r] & 0
    }
  \]

  Where there are $2n-1$ copies of $\bb{Z}$ in the above complex.
  We also know that the cellular cochain complex is isomorphic to the hom dual
  of the cellular cochain complex. Moreover $\Hom(\bb{Z},A)\cong A$, determined by
  where the generators are sent, and $\Hom(\bb{Z},-)(\cdot m)=\cdot m$. We know look
  at the cellular cochain complex for each of the listed groups in turn.

  \begin{itemize}
  \item[$\bb{Z}$:] From the above information we know that cellular cochain
    complex for $\bb{Z}$ will in fact be almost exactly the same as the cellular chain complex
    \[
      \xymatrix{
        0 \ar[r] & \bb{Z} \ar[r]^0 & \bb{Z} \ar[r]^m & \bb{Z} \ar[r]^0 & \cdots \ar[r]^0 & \bb{Z} \ar[r]^m & \bb{Z} \ar[r]^0 & \bb{Z} \ar[r] & 0
      }
    \]
    Thus the cohomology groups of the lens space with integer coefficients
    will be isomorphic to the homology groups with a switch from $k$ odd to $k$ even
    \[
      H^k(L_m(\ell_1,\ldots,\ell_n);\bb{Z}) \cong  
      \left\{
        \begin{array}{ll}
          \bb{Z}& k=0,2n-1\\
          \bb{Z}_m& 0<k<2n-1,\ \text{$k$ even}\\
          0 & \text{otherwise}
        \end{array}
      \right.
    \]
  \item[$\bb{Q}$:] With rational coefficients the cellular cochain complex will
    be of the form
    \[
      \xymatrix{
        0 \ar[r] & \bb{Q} \ar[r]^0 & \bb{Q} \ar[r]^m & \bb{Q} \ar[r]^0 & \cdots \ar[r]^0 & \bb{Q} \ar[r]^m & \bb{Q} \ar[r]^0 & \bb{Q} \ar[r] & 0
      }
    \]

    It's clear from this that for $k=2n-1,0$ that the cohomology will be $\bb{Q}$. The
    other cases are either an $\cdot m$ leaving and a $0$ map entering, or vice versa.
    In the prior case the kernel is trivial and in the latter case the image equals the
    kernel. Either way the cohomology will be zero. Thus the cohomology for the lens
    space with rational coefficients will be
    \[
      H^k(L_m(\ell_1,\ldots,\ell_n);\bb{Q}) \cong  
      \left\{
        \begin{array}{ll}
          \bb{Q} & k=2n-1,0\\
          0 & \text{otherwise}
        \end{array}
      \right.
    \]
  \item[$\bb{Z}_m$:] Under $\bb{Z}_m$ the map that multiplies by $m$ gives us
    the zero map. As such the cellular cochain complex will be
    \[
      \xymatrix{
        0 \ar[r] & \bb{Z}_m \ar[r]^0 & \bb{Z}_m \ar[r]^0 & \bb{Z}_m \ar[r]^0 & \cdots \ar[r]^0 & \bb{Z}_m \ar[r]^0 & \bb{Z}_m \ar[r]^0 & \bb{Z}_m \ar[r] & 0
      }
    \]
    Since the kernel is always $\bb{Z}_m$ and the image the identity we then have that

    \[
      H^k(L_m(\ell_1,\ldots,\ell_n);\bb{Z}_m) \cong  
      \left\{
        \begin{array}{ll}
          \bb{Z}_m & 0\leq k < 2n-1\\
          0 & \text{otherwise}
        \end{array}
      \right.
    \]
  \item[$\bb{Z}_p$:]
    Dualizing with $\bb{Z}_p$ gives us the cellular cochain complex
    \[
      \xymatrix{
        0 \ar[r] & \bb{Z}_p \ar[r]^0 & \bb{Z}_p \ar[r]^0 & \bb{Z}_p \ar[r]^0 & \cdots \ar[r]^0 & \bb{Z}_p \ar[r]^0 & \bb{Z}_p \ar[r]^0 & \bb{Z}_p \ar[r] & 0
      }
    \]

    Then we need to consider the case where $p|m$ and when $p\nmid m$.

    When $p|m$ the map $\cdot m$ becomes the zero map. As such we end up
    with cohomology analagous to the $\bb{Z}_m$ case

    \[
      H^k(L_m(\ell_1,\ldots,\ell_n);\bb{Z}_p) \cong  
      \left\{
        \begin{array}{ll}
          \bb{Z}_p & 0\leq k<2n-1\\
          0 & \text{otherwise}
        \end{array}
      \right.
    \]

    However if $p\nmid m$. Then the map $\cdot m$ is invertible and as such an
    isomorphism. This makes it analagous to the $\bb{Q}$ case

    \[
      H^k(L_m(\ell_1,\ldots,\ell_n);\bb{Z}_p) \cong  
      \left\{
        \begin{array}{ll}
          \bb{Z}_p & k=0,2n-1\\
          0 & \text{otherwise}
        \end{array}
      \right.
    \]
  \end{itemize}

  Now we check our work with the universal coefficient theorem. The
  short exact sequence that we will be using is
  \[
    \xymatrix{
      0 \ar[r] & \Ext(H_k(L),G) \ar[r] & H^k(L;G) \ar[r] & \Hom(H_k(L),G) \ar[r] & 0
    }
  \]

  If $k=2n-1,0$ we get
  \[
    \xymatrix{
      0 \ar[r] & \Ext(0,G)=0 \ar[r] & H^k(L;G) \ar[r] & \Hom(\bb{Z},G)=\bb{Z} \ar[r] & 0
    }
  \]
  which provides $H^{2n-1}(L;G)\cong H^0(L;G)\cong G$ as expected.

  Next if $0<k<2n-1$ and $k$ even we break into cases:
  \begin{itemize}
  \item[$\bb{Z}$:]
  \item[$\bb{Q}$:]
  \item[$\bb{Z}_m$:]
  \item[$\bb{Z}_p$:]
  \end{itemize}

  Finally if $0<k<2n-1$ and $k$ odd we break into the same cases as before:
  \begin{itemize}
  \item[$\bb{Z}$:]
  \item[$\bb{Q}$:]
  \item[$\bb{Z}_m$:]
  \item[$\bb{Z}_p$:]
  \end{itemize}
\end{proof}

\sk

\begin{problem}[3.1.11]
  Let $X$ be a Moore space $M(\bb{Z}_m,n)$ obtained from $S^n$ by attaching
  a cell $e^{n+1}$ by a map of degree $m$.
  \begin{itemize}
  \item[(a)] Show that the quotient map $X\rightarrow X/S^n=S^{n+1}$ induces the
    trivial map on $\rH_i(-;\bb{Z})$ for all $i$, but not on
    $H^{n+1}(-;\bb{Z})$. Deduce that the splitting in the universal coefficient
    theorem for cohomology cannot be natural.
  \item[(b)] Show that the inclusion $S^n\hookrightarrow X$ induces the trivial
    map on $\rH^i(-;\bb{Z})$ for all $i$, but not on $H_n(-;\bb{Z})$.
  \end{itemize}
\end{problem}

\begin{proof}
  \begin{itemize}
  \item[(a)]  We can deduce that the map induced by $q:X\rightarrow S^{n+1}$ is the trivial on homology
    since $\rH_k(X)\cong 0$ for $k\neq n$ implies that it can only be the zero map and for
    $k=n$ we have $q_*:\bb{Z}_m\rightarrow (H_n(S^{n+1})=0)$.

    Using the universal coefficient theorem we can calculate the cohomology of
    $X$ with $\bb{Z}$ coefficients. It is clear that $\rH^k(X;\bb{Z})\cong 0$ when
    $k\neq n+1$ as both $\Hom$ and $\Ext$ will be zero in the universal coefficient
    theorem. However when $k=n+1$
    \[
      \xymatrix{
        0 \ar[r] & \Ext(H_{n}(X)\cong \bb{Z}_m,\bb{Z})\cong \bb{Z}_m \ar[r] & H^{n+1}(X) \ar[r] & \Hom(H_{n+1}(X)\cong 0,\bb{Z})\cong 0 \ar[r] & 0\\
      }
    \]
    Giving us that $H^{n+1}(X)\cong \bb{Z}_m$. From the same reasoning as before we know
    that the induced map $q^*:\rH^k(S^{n+1})\rightarrow\rH^k(X;\bb{Z})$ will be the zero map
    for $k\neq n+1$. When $k=n+1$ if we look at $q_{\#}$ on the chain level it
    will send the $e^{n+1}$ cell of $X$ to the $e^{n+1}$ cell for $S^{n+1}$. As
    such when we look at the induced map $q^*:\rH^{n+1}(S^{n+1})=\bb{Z}\rightarrow \rH^{n+1}(X)=\bb{Z}_m$
    it will send the generator of the former to the generator of the latter. As
    such the map $q^*$ is not trivial on cohomology when $k=n$.

    \textbf{Say why this shows the universal coefficient theorem does not split
    naturally.}
\item[(b)] For the same reason as $q_*$ being trivial on reduced homology above, $i^*$
  will be trivial on reduced cohomology since there are no groups that are nonzero
  for the same index.

  However when it comes to reduced homology when $k=n$ the induced map $i_*$
  will map the generator of $\rH_n(S^n)=\bb{Z}$ to the generator of the reduced
  homology of $X$ which is also a copy of $S^n$. As such $i_*(k)=k\mod m$
  on the $n$th homology group.
  \end{itemize}
\end{proof}

\end{document}
%%%%%%%%%%%%%%%%%%%%%%%%%%%%%%%%%%%%%%%%%%%%%%%%%%%%%%%%%%%%%%%%%%%%%%%%%%%%%%%% 
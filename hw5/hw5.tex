\documentclass[10pt]{article}
\usepackage[utf8]{inputenc}
\usepackage{amscd}
\usepackage{amsmath}
\usepackage{amssymb}
\usepackage{amsthm}
\usepackage{listings}
\usepackage{enumerate}
\usepackage[all,cmtip]{xy}

\textwidth=15cm \textheight=22cm \topmargin=0.5cm \oddsidemargin=0.5cm \evensidemargin=0.5cm

\newcommand{\sk}{\vskip 10mm}
\newcommand{\bb}[1]{\mathbb{#1}}
\newcommand{\ra}{\rightarrow}
\newcommand{\rH}{\widetilde{H}}
\DeclareMathOperator{\img}{Im}
\DeclareMathOperator{\Ext}{Ext}
\DeclareMathOperator{\Hom}{Hom}

\theoremstyle{remark}
\newtheorem{problem}{Problem}
\newtheorem{lemma}{Lemma}[problem]


\newtheorem{tpart}{}[problem]
\newtheorem*{ppart}{}

\begin{document}

\begin{problem}[3.2.3]
  \begin{enumerate}
  \item[(a)] Using the cup product structure, show there is no map
    $\bb{R}P^n\rightarrow\bb{R}P^m$ inducing a nontrivial map
    $H^1(\bb{R}P^m;\bb{Z}_2)\rightarrow H^1(\bb{R}P^n;\bb{Z}_2)$ if $n>m$. What
    is the corresponding results for maps $\bb{C}P^n\rightarrow\bb{C}P^m$?
  \item[(b)] Prove the Borsuk-Ulam theorem by the following argument. Suppose on
    the contrary that $f:S^n\rightarrow \bb{R}^n$ satisfies $f(x)\neq f(-x)$ for
    all $x$. Then define $g:S^n\rightarrow S^{n-1}$ by
    $g(x)=(f(x)-f(-x))/|f(x)-f(-x)|$, so $g(-x)=-g(x)$
    and $g$ induces a map $\bb{R}P^n\rightarrow\bb{R}P^{n-1}$. Show that
    part (a) applies to this map.
  \end{enumerate}
\end{problem}

\begin{proof}
  
\end{proof}

\sk

\begin{problem}[3.2.5]
  Show the ring $H^*(\bb{R}P^\infty;\bb{Z}_{2k})$ is isomorphic to
  $\bb{Z}_{2k}[\alpha,\beta]/(2\alpha,2\beta,\alpha^2-k\beta)$ where
  $|\alpha|=1$ and $|\beta|=2$. [Use the coefficient map
  $\bb{Z}_{2k}\rightarrow\bb{Z}_2$ and the proof of Theorem 3.19.]
\end{problem}

\begin{proof}
  
\end{proof}

\sk

\begin{problem}[3.2.7]
  Use the cup products to show that $\bb{R}P^3$ is  not homotopy equivalent to
  $\bb{R}P^2\vee S^3$.
\end{problem}

\begin{proof}
  From Hatcher the cup product structure for $\bb{R}P^3$ with $\bb{Z}_2$ coefficients
  is
  \[
    H^*(\bb{R}P^3;\bb{Z}_2)\cong \bb{Z}_2[\alpha,\beta]/(\alpha^4)\quad |\alpha|=1
  \]
  and the cup product structure for $\bb{R}P^2\vee S^3$ with $\bb{Z}_2$ coefficients is
  \[
    H^*(\bb{R}P^2\vee S^3;\bb{Z}_2)\cong H^*(\bb{R}P^2;\bb{Z}_2)\times H^*(S^3;\bb{Z}_2)
    \cong(\bb{Z}_2[\beta]/(\beta^3)\times\bb{Z}_2[\gamma]/(\gamma^2))/(\langle 1_\beta,1_\gamma\rangle)\quad |\beta|=1,|\gamma|=3
  \]
  The latter has an element, $\beta$, which when cubed is zero. However the former
  has no such elements. Thus the two cohomology rings are not isomorphic
  and therefore $\bb{R}P^3$ is not homotopy equivalent to $\bb{R}P^2\vee S^3$.
\end{proof}

\end{document}
%%%%%%%%%%%%%%%%%%%%%%%%%%%%%%%%%%%%%%%%%%%%%%%%%%%%%%%%%%%%%%%%%%%%%%%%%%%%%%%%